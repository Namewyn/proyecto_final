\chapter*{}
%\thispagestyle{empty}
%\cleardoublepage

%\thispagestyle{empty}

%\input{portada/portada_2}



\cleardoublepage
\thispagestyle{empty}

\begin{center}
{\large\bfseries Daño Axonal Difuso: La marca de los traumatismos craneoencefálicos}\\
\end{center}
\begin{center}
Adela Sabio González \\
\end{center}

%\vspace{0.7cm}
\noindent{\textbf{Palabras clave}: proteína precursora de $\beta$ amiloide daño axonal difuso, inmunohistoquímica, tiempo de supervivencia, medicina forense}\\

\vspace{0.7cm}
\noindent{\textbf{Resumen}}\\

URL: \url{https://github.com/Namewyn/proyecto_final}


El daño axonal difuso es uno de los principales mecanismos por el que se produce la muerte en casos de traumatismos craneoencefálicos. Su evaluación histológica, según la técnica empleada, sólo es posible a partir de las 2 horas de supervivencia. El diagnóstico de daño axonal difuso puede ser relevante en la práctica de la medicina forense. Se han realizado estudios previos que intentan asociar patrones de lesión empleando anticuerpos contra el precursor de la proteína $\beta$-amiloide, con el fin de establecer una correlación entre los mismos y la etiología de la muerte. Este trabajo se enmarca como estudio preliminar para la realización de un proyecto en el que se pretende analizar la prevalencia de daño axonal difuso en distintas patologías de origen traumático. Se han tomado 11 muestras de encéfalo procedentes de individuos con distintas patologías (traumáticas y no traumáticas) para preparar y evaluar el funcionamiento del anticuerpo contra $\beta$-APP, así como para observar distintos patrones de lesión axonal no necesariamente vinculadas a una contusión. De los 11 casos, 2 mostraron rasgos de daño axonal difuso, 4 mostraron lesiones isquémicas, 1 mostró otras alteraciones y el resto fueron negativas. Los hallazgos fueron concordantes con la historia previa de la que se disponía. 
\cleardoublepage


\thispagestyle{empty}


\begin{center}
{\large\bfseries Diffuse axonal injury}\\
\end{center}
\begin{center}
Adela, Sabio\\
\end{center}

%\vspace{0.7cm}
\noindent{\textbf{Keywords}: $\beta$ precursor protein, diffuse traumatic axonal injury, inmmunohistochemistry, survival time,head injuries, forensic medicine}\\

\vspace{0.7cm}
\noindent{\textbf{Abstract}}\\

URL: \url{https://github.com/Namewyn/proyecto_final}

Diffuse axonal injury is one of the main mechanisms of death in cases of traumatic head/brain injury. Histological assessment is possible as early as 2 hours after the initial injury depending on the stain it’s used. The diagnosis of diffuse axonal injury may be of considerable importance in forensic medicine. There are earlier studies which tries, using immunohistochemistry with antibodies against $\beta$-amyloid precursor protein, to associate different morphological patterns and distributions of $\beta$-APP immunoreactive axons with distinct clinical entities. This work is a preliminary study of a project with the aim to analyse prevalence of diffuse axonal injury in several pathologies of traumatic origin. The case material consisted of 11 cases of a range of diseases (traumatic and non traumatic) to prepare and evaluate antibodies against $\beta$-amyloid precursor protein and to assess different patterns of axonal injury. 2/11 cases showed diffuse axonal injury, 4/11 cases ischemic lesions, 1/11 cases other abnormalities and 5/11 cases were negative. These results were concordant with the previosus case history.

\chapter*{}
\thispagestyle{empty}

\noindent\rule[-1ex]{\textwidth}{2pt}\\[4.5ex]

Yo, \textbf{Adela Sabio González}, alumno del \textbf{Máster en Antropología Física y Forense} de la \textbf{Universidad de Granada}, con DNI 15432279S, autorizo la
ubicación de la siguiente copia de mi Trabajo Fin de Máster en la biblioteca del centro para que pueda ser
consultada por las personas que lo deseen.

\vspace{6cm}

\noindent Fdo: Adela Sabio González

\vspace{2cm}

\begin{flushright}
Granada a 22 de mes octubre de 2018 .
\end{flushright}


\chapter*{}
\thispagestyle{empty}

\noindent\rule[-1ex]{\textwidth}{2pt}\\[4.5ex]

Dra. \textbf{María Inmaculada Alemán Aguilera}, Profesor del Área de Antropología del Departamento Antropología Física y Forense de la Universidad de Granada.

\vspace{0.5cm}

Dra. \textbf{Elisa María Cabrerizo Medina}, Médico Forense del Instituto de Medicina Legal y Ciencias Forenses de Granada.


\vspace{0.5cm}

\textbf{Informan:}

\vspace{0.5cm}

Que el presente trabajo, titulado \textit{\textbf{Daño Axonal Difuso: la marca de los traumatismos craneoencefálicos}},
ha sido realizado bajo su supervisión por \textbf{Adela Sabio González}, y autorizamos la defensa de dicho trabajo ante el tribunal
que corresponda.

\vspace{0.5cm}

Y para que conste, expiden y firman el presente informe en Granada a 22 de octubre de 2018.

\vspace{1cm}

\textbf{Los directores:}

\vspace{5cm}

\noindent \textbf{María Inmaculada Alemán Aguilera \ \ \ \ \ Elisa María Cabrerizo Medina}


\chapter*{}
\thispagestyle{empty}

\noindent\rule[-1ex]{\textwidth}{2pt}\\[4.5ex]

Yo, \textbf{Adela Sabio González}, alumno del \textbf{Máster en Antropología Física y Forense} de la \textbf{Facultad de Medicina de la Universidad de Granada}, con DNI 15432279S, \underline{declaro explícitamente} que el trabajo presentado es original, entendido en el sentido que no he utilizado ninguna fuente sin citarla debidamente.

\vspace{6cm}

\noindent Fdo: Adela Sabio González

\vspace{2cm}

\begin{flushright}
Granada a 22 de mes octubre de 2018.
\end{flushright}


\chapter*{Agradecimientos}
\thispagestyle{empty}

       \vspace{1cm}


A José, por acompañarme y apoyarme en todo lo que emprendo.

A nuestro Departamento de Anatomía Patológica, sobre todo a Ovidiu, por su paciencia y ayuda para poner en marcha el anticuerpo en el que se basa el estudio, a Rosa, por cortar todas mis muestras, a Antonio y Miriam, por su ayuda con la inmuno, a Titi, por colarme las preparaciones en el escáner y a Lucía, por facilitarme el día a día, permitiéndome sacar tiempo para realizar este proyecto.

A mi tutora, Elisa, por su gran disposición para ayudarme a plantear el trabajo y conseguir muestras. 

Y muy especialmente a Rosa, mi tutora de residencia, por su disponibilidad constante ante cualquier dificultad, por todo el tiempo que me ha dedicado y por adoptarme cuando no sabía por dónde empezar. 


